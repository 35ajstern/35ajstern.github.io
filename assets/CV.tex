%%%%%%%%%%%%%%%%%%%%%%%%%%%%%%%%%%%%%%%%%
% Medium Length Graduate Curriculum Vitae
% LaTeX Template
% Version 1.1 (9/12/12)
%
% This template has been downloaded from:
% http://www.LaTeXTemplates.com
%
% Original author:
% Rensselaer Polytechnic Institute (http://www.rpi.edu/dept/arc/training/latex/resumes/)
%
% Important note:
% This template requires the res.cls file to be in the same directory as the
% .tex file. The res.cls file provides the resume style used for structuring the
% document.
%
%%%%%%%%%%%%%%%%%%%%%%%%%%%%%%%%%%%%%%%%%

%----------------------------------------------------------------------------------------
%	PACKAGES AND OTHER DOCUMENT CONFIGURATIONS
%----------------------------------------------------------------------------------------

\documentclass[margin, 10pt]{res} % Use the res.cls style, the font size can be changed to 11pt or 12pt here

\usepackage{helvet} % Default font is the helvetica postscript font
\usepackage{hyperref}
%\usepackage{newcent} % To change the default font to the new century schoolbook postscript font uncomment this line and comment the one above

\setlength{\textwidth}{5.1in} % Text width of the document

\begin{document}

%----------------------------------------------------------------------------------------
%	NAME AND ADDRESS SECTION
%----------------------------------------------------------------------------------------

\moveleft.5\hoffset\centerline{\large\bf Aaron J. Stern} % Your name at the top
 
\moveleft\hoffset\vbox{\hrule width\resumewidth height 1pt}\smallskip % Horizontal line after name; adjust line thickness by changing the '1pt'
 
\moveleft.5\hoffset\centerline{2711 California St., Apt. B} % Your address
\moveleft.5\hoffset\centerline{Berkeley, CA 94703}
\moveleft.5\hoffset\centerline{(415) 342-1815}
\moveleft.5\hoffset\centerline{Email: \url{ajstern@berkeley.edu}}
\moveleft.5\hoffset\centerline{Github: \texttt{35ajstern}}
\moveleft.5\hoffset\centerline{Personal website: \url{35ajstern.github.io}}
\moveleft.5\hoffset\centerline{Twitter: \texttt{\makeatletter @\makeatother aaron\_\_stern}}

%----------------------------------------------------------------------------------------

\begin{resume}

%----------------------------------------------------------------------------------------
%	OBJECTIVE SECTION
%----------------------------------------------------------------------------------------
 
\section{EDUCATION}  
{\bf Ph.D. Computational Biology}\hfill 2015 - present\\
{\sl UC Berkeley}\\
{\sl Topic:} I am interested in theoretical population and statistical genetics, with a focus on modeling how natural selection shapes the genetic architecture of complex traits. I apply these models to human genomic data to characterize the traits and parts of our genetic architecture which underly adaptation. I am also interested in how natural selection impacts standard genetic epidemiological methods.\vspace{0.1cm}\\
Advised by Prof. Rasmus Nielsen, Depts. of Statistics \& Integrative Biology.\\
Dissertation committee: Rasmus Nielsen, Yun S. Song, Priya Moorjani, Ian J. Wang

{\bf B.Sc. Engineering}\hfill 2011 - 2015\\
{\sl Northwestern University}\\
Advised by Prof. Luis Amaral, Dept. of Chemical \& Biological Engineering

\section{TEACHING EXPERIENCE}
 Stat 135 {\sl Concepts of Statistics}  \hfill Fall 2018\\
UC Berkeley, Berkeley, CA\\
\begin{itemize}
\item[] Led discussion sections and held office hours. The course, taught by Adam Lucas, surveys the theory and practice (using R) of applied statistics, aimed at an audience of junior/senior statistics majors. 
\item[] Received Outstanding Graduate Student Instructor award (awarded to $\sim$9\% of GSIs)
\end{itemize} 

{\sl Instructor}, Python Bootcamp \hfill 2018 - 2019 \\
Center for Computational Biology, UC Berkeley

\section{WORK EXPERIENCE}

{\sl Computational Biology Fellow} \hfill June 2017 - Sept 2017, March 2018 - July 2018 \\
AncestryDNA, San Francisco, CA
\begin{itemize}
\item[] Developed methods and visualization tools to explore the recent demographic history of fine-scale human communities. I also worked on statistical methods for relationship and pedigree inference. 
\end{itemize} 

{\sl Research Intern} \hfill Sept 2014 - June 2015 \\
Amaral Lab, Northwestern University
\begin{itemize}
\item[] I studied variation in the Shine-Dalgarno (SD) sequence, a short translational motif widespread in bacteria yet uncharacterized for all but a small number of species. I developed statistical methods for quantifying SD usage. I applied these methods to show a novel correlation between SD usage and species growth rate and doubling time.
\end{itemize} 

{\sl Research \& Development Intern} \hfill June 2014 - November 2014\\
BigDataBio, LLC, Cambridge, MA
\begin{itemize}
\item[] Deployed and wrote documentation for a visualization software pipeline to take .VCF and visualize the sample’s variants and their implications in disease.
\end{itemize} 


\section{SKILLS} 

{\sl Programming languages:} Python, UNIX, \LaTeX\\
{\sl Selected courses:} Statistical Learning Theory, Applied Probability, Applied Statistics, Theoretical Statistics, Information Theory, Real Analysis, Parallel Programming, Protein Structure \& Function, Computational Biology, Population Genetics

\section{MENTORSHIP}

Yulin Zhang (2019 -- present) --- senior at Sun Yat-sen University in Guangzhou, China. \vspace{0.1cm}\\
Courtney Rauchman (2018 -- present) --- is in her fourth year of undergraduate studies at UC Berkeley, majoring in Data Science. She received a \$5,000 SURF award for our research.
 
%----------------------------------------------------------------------------------------
%	PROFESSIONAL EXPERIENCE SECTION
%----------------------------------------------------------------------------------------



%----------------------------------------------------------------------------------------
%	COMMUNITY SERVICE SECTION
%---------------------------------------------------------------------------------------- 

\section{VOLUNTEERING}

Bay Area Population Genetics Meeting (Fall 2019)\\
\begin{itemize}
\item[] Lead organizer \& master of ceremonies
\end{itemize} 

Bay Area Scientists in Schools (BASIS) \hfill Sept 2015 - present\\
{\sl Team Leader}
\begin{itemize}
\item[] STEM enrichment for elementary school students in local Title I schools. I design and lead interactive sci- entific demonstrations relating to natural selection and speciation
\end{itemize} 
Be A Scientist \hfill Sept 2015 - present\\
{\sl Volunteer}
\begin{itemize}
\item[] STEM mentorship for 7th-graders in a local Title I public school. I guide a small group of students in designing, deploying, and interpreting their own scientific experiments
\end{itemize} 
UC Berkeley Center for Computational Biology \hfill Sept 2015 - present\\
{\sl Volunteer}
\begin{itemize}
\item[-] Organized guest faculty speaker series (2016-17) with equal gender representation
\item[-] Organized student research symposium (2016)
\item[-] Organized student research lunchtime seminars (2017-18)
\item[-] Volunteered for Berkeley Connect (career \& mentorship group) as panelist and industry visit chaperone
\end{itemize}
Summerbridge SF \hfill 2008 - 2011\\
{\sl Volunteer tutor}
\begin{itemize}
\item[] Mentored and tutored lower-income middle school students in mathematics through a tuition-free academic enrichment program.
\end{itemize}

\section{PUBLICATIONS}

Cheng JY, {\bf Stern AJ}, Racimo F, Nielsen R, ``Ohana: detecting selection in multiple populations by
modelling ancestral admixture components'', under review at {\it Molecular Biology \& Evolution} (2019).

{\bf Stern AJ}, Wilton PR, Nielsen R, ``An approximate full-likelihood method for inferring selection and allele frequency trajectories using DNA sequence data'', {\it PLOS Genetics} (2019).

{\bf Stern AJ}, Nielsen R, ``Detecting Natural Selection,'' {\it Handbook of Statistical Genetics, 4th Edition},  John Wiley \& Sons 2019 (in press).

Prohaska A$^\dag$, Racimo F$^\dag$, Schork AJ$^\dag$, Sikora M, {\bf Stern AJ}*, Ilardo M*, {\it et al.}, ``Human Disease Variation in Light of Population Genomics,'',  {\it Cell Reviews} (2019).

Corl A, Bi K, Luke C, Challa AS, {\bf Stern AJ}, Sinervo B, Nielsen R, ``Genetic change follows plasticity for cryptic coloration in a novel environment'', {\it Current Biology} (2018).

Ilardo M, Moltke I, Korneliussen T, Cheng J, {\bf Stern AJ}, {\it et al}, ``Physiological and genetic adaptations to diving in Sea Nomads'', {\it Cell} (2018)

Hockenberry AJ, {\bf Stern AJ}, Amaral LAN, Jewett M,``Bacterial growth rate demands shape variation,'' {\it Molecular Biology \& Evolution} (2017).

Gong B*, {\bf Stern AJ*}, Wang Y*, Zhou Y*, ``Review of Statistical Analysis of Numerical Preclinical Radiobiological Data'', {\it ScienceOpen Research} (2016). 


* {\it indicates equal contribution}\\
$\ddag$ {\it indicates mentored student}


\section{INVITED\\TALKS}
{\bf Stern AJ}, Nielsen R, ``An approximate full-likelihood method for detecting selection using ancestral recombination graphs,'' Bay Area Population Genetics, November 4, 2017.\\
{\bf Stern AJ}, Nielsen R, ``An approximate full-likelihood method for detecting selection using ancestral recombination graphs'' American Society of Human Genetics, October 19, 2017.

\section{ABSTRACTS}
{\bf Rauchman C}$^{\ddag}$, {\bf Stern AJ}*, Nielsen R, ``Selection biases LD-based heritability estimation,'' Bay Area Population Genetics, November 9, 2019.\\
{\bf Stern AJ}, Speidel L, Zaitlen NA, Nielsen R, ``Disentangling selection on genetically-correlated traits using population genomic data,'' American Society of Human Genetics, October 16, 2019.\\
{\bf Rauchman C}$^{\ddag}$, {\bf Stern AJ}*, Nielsen R, ``Selection biases LD-based heritability estimation,'' American Society of Human Genetics, October 16, 2019.\\
Song S, {\bf Stern AJ}, {\it et al}, ``Studying global variation of gene flow using geo-referenced genetic data,'' American Society of Human Genetics, October 19, 2017.\\
{\bf Stern AJ}, Nielsen R, ``An approximate full-likelihood method for detecting selection using ancestral recombination graphs,'' Society of Molecular Biology and Evolution, 2017.\\
{\bf Stern AJ}, Hockenberry AJ,  Amaral LAN, “Rationally defining Shine-Dalgarno motifs in bacteria,” Center for Computational Biology Retreat, November 21, 2015.

\section{AWARDS} 

Outstanding Graduate Student Instructor, Dept. of Statistics \hfill 2018\\
Center for Computational Biology Student Travel Grant (\$1630) \hfill 2017\\
Dean’s Scholar, School of Engineering, Northwestern University \hfill 2011 - 2015\\

%----------------------------------------------------------------------------------------

\end{resume}
\end{document}